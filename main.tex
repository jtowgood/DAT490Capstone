\documentclass{article}
\usepackage{graphicx} % Required for inserting images

\usepackage{amsmath}
\usepackage{graphicx}
\usepackage[colorlinks=true, allcolors=blue]{hyperref}
\usepackage{authblk}
\usepackage{fullpage}
\usepackage{hyperref}
\usepackage{indentfirst}

\title{\textbf{Literature Review:} Analyzing Liquor Sales to Provide Insight into Lifestyle Changes in Iowa}

\author[1]{Anthony Rosamond}
\author[1]{Madison Brown}
\author[1]{Jade Towgood}
\author[1]{Jasmine Gomez}

\affil[1]{Arizona State University, Tempe, AZ 85281, USA}

\date{September 13, 2024}

\begin{document}

\maketitle

\section{Introduction}

Liquor plays a significant role within both social and economic matters, especially in the state of Iowa. Resident’s high alcohol consumption has caused the state to work towards addressing public health concerns related to excessive drinking. However, as alcohol sales contribute substantially to the economy, it is not possible to outright ban the sale of liquor fiscally or socially. Historically, outright bans such as during the Prohibition era failed dramatically. Instead, a balance between the sale of alcohol and ensuring public health and safety is of upmost importance. 

For the case of Iowa, studying liquor sales across the different regions of the state can highlight areas of public concern, potentially through identification of unusually high rates of liquor sales. Analyzing the types and quantities of alcohol being sold can also reveal changes in consumer preferences or highlight the impact of state regulations, with each factor potentially influencing the other. The liquor sales transactions dataset \cite{liquor-sales} also provides an opportunity of developing insights towards predicting future liquor sales, a retail strategy that has broad applications across a variety of industries. 

What makes liquor sales interesting to study, and the aim of our research, is the complex and unique intersection of consumer behavior, retail sales, and public health that go beyond typical retail research. The focus of this project is to investigate historical liquor sales in Iowa to track changes in consumer preferences across generations, neighborhoods, and individual retail markets with particular attention to areas of high liquor sales as well as how new policies have influenced this. 

\section{Background}
\subsection{Relevant History and Context}
Alcoholic beverages have been around for thousands of years. Within the United States, there are many regions where alcohol consumption is very common amongst various populations. However, despite alcohol's long history, current attitudes and laws regarding alcohol sales and consumption within the United States are not aligned with the rest of the world's standards. There have been multiple moments where either individual states or the federal government have imposed severe restrictions on the sales and consumption alcohol and the state of Iowa has been no exception. 

Throughout the history of the United States there have been many political issues where decisions on regulations were made at the state level. Alcohol was one of those issues and to a certain extent still falls under state authority to this day. After Iowa was admitted as a state in 1846, there had been public campaigns in order to regulate alcohol. This was spearheaded by the Women's Christian Temperance Union of Iowa which led to members of the Iowa Republican Party to introduce and pass a state amendment imposing a total ban on alcohol \cite{kutz2023} in 1882. However, this would eventually be dismissed by the state's supreme court. Alcohol remained tightly regulated within Iowa  until 1920 when the Eighteenth Amendment to the US Constitution was passed, which banned the production, transportation, and sale of alcohol nationwide, also know as the prohibition. This would remain in effect until the Twenty-First Amendment was passed in 1933 to repeal the Eighteenth. By 1966, all states had repealed their prohibition laws \cite{hist2009}. 

Now, the modern alcohol industry in the state of Iowa is booming with 100 breweries, 120 wineries and 45 distilleries as of 2023 \cite{kutz2023}. However that does not necessarily mean that everything is all well when it comes to alcohol. Excessive drinking has historically been more predominant in the Midwestern states, with Iowa ranking number 4 when it comes to binge drinking \cite{axios2024}. As binge drinking is directly linked to health issues such as chronic disease and compromised immune systems, this has the potential to become a larger public health issue \cite{axios2024}. Part of the state of Iowa's focus on addressing alcohol consumption has led to detailed publication of liquor sale transactions across the entire state \cite{iowa_liquor_sales_data}. 

\subsection{Key Sources/Papers}
In order to analyze liquor sales trends in Iowa and their broader implications, several key sources are necessary to provide valuable insights. The Iowa Liquor Sales Data from \url{data.iowa.gov} is an essential tool for understanding alcohol consumption patterns across various regions, offering a detailed view of how factors like urbanization and economic conditions influence liquor sales \cite{iowa_liquor_sales_data}. Similarly, the Iowa Enterprise Data Liquor Sales Snapshot provides real-time, monthly data from January 2012 through August 2024 that tracks sales by product type and region \cite{iowa_liquor_sales_snapshot}. This dataset is valuable for analyzing seasonal shifts in drinking habits, especially around holidays or local events.

Context on external factors affecting alcohol consumption is provided by Columbia University’s study on liquor sales during COVID-19, which focuses on the pandemic’s impact on the move toward home consumption \cite{columbia_study}. This study highlights how external crises can reshape consumer behaviors. Harvard Public Health’s article on the "Sober Curious" movement adds another dimension, emphasizing how health concerns and changing social values, particularly among younger consumers, are driving new trends in alcohol consumption \cite{harvard_sober_curious}.

Additionally, Penn State Extension’s report on Alcoholic Beverage Consumption and Purchasing Trends sheds light on how consumer preferences are evolving in the broader alcohol market, which may parallel trends in Iowa \cite{penn_state_extension}. Finally, the Iowa Sales Data Revisited study provides a historical perspective on how changes in Iowa’s liquor control laws have affected consumption, offering important context for long-term trends in the state’s drinking habits \cite{iowa_sales_revisited}.

Together, these sources help build a comprehensive picture of how liquor sales trends in Iowa reflect broader changes in population dynamics, lifestyle choices, and societal behaviors.

\subsection{Key Controversies}
An essential concept within the study of alcohol sales and consumption is that of alcohol outlet density, which measures the numbers of physical locations wherein alcoholic beverage are sold legally per population \cite{campbell2009effectiveness}. One major assumption within current alcohol regulations is that the easier it is for a population to access alcohol, the greater the amount of alcohol consumed and subsequent alcohol-related issues \cite{babor2010alcohol}. However, these studies that involve studying sub populations and their alcohol consumption patters utilize Census data to identify important regions of focus. There is a major controversy surrounding the use of United States Census demographic data, but despite that it continues to be the typical source for demographic data when investigating alcohol consumption and/or its related public health issues. Oftentimes, the United States Census under counts minority subgroups, including poor or racial minorities, which can lead to inaccurate counts of sub populations within various regions \cite{anderson2000race}. Yet, Census data is typically used to identify areas of high risk. For instance, multiple sub populations have been identified as being at risk for some form of alcohol abuse, such as Native Americans and Blacks, but these sub populations and their neighborhoods are determined using Census data \cite{delker2016alcohol}. Studies that focus on the overlap of alcohol outlet density, racial demographics, age, or sex using the Census to provide data on these demographics can be skewed due to under count of minorities. Therefore, policies or regulations based on these studies may not have an accurate demographic picture on the regions they are affecting. 

Other debated influential factors on alcohol sales include the effect of societal standards such as alcohol marketing and the proximity of alcohol outlets to each other or key community areas such as schools \cite{campbell2009effectiveness, islam2022usefulness, wilcox2015beer}. Bans on alcohol advertising, in school areas particularly, are e major source of debate. This is due to concern over increasing exposure of alcohol to youths and adolescents, with the primary belief being that it will encourage underage drinking \cite{wilcox2015beer}. One of the most recent studies examining the effect of alcohol advertising on consumption within the United States over a 40-year period found that alcohol advertising does not influence alcohol consumption, in direct contrast to a common government method of regulating alcohol \cite{wilcox2015beer, babor2010alcohol}. However, this same study did find that the sales of individual categories of alcohol (beer, wine, or liquor) can be influenced by outside societal factors, which marketing can play into \cite{wilcox2015beer}. Thus, it is a highly debated factor that may influence alcohol sales and its regulations. 

\subsection{Open Ended Questions}
After taking into account all the information that was gathered through research, we are still left with some issues that deserve further exploration. Researching this topic raises questions on if and how Iowa’s demographics ties into alcohol consumption, as well as any influence from income. Another question that came up is how liquor sales over the years have changed in relation to the implementation of different regulatory policies in Iowa. The goal of this question is to determine if these new changes are creating the desired effect on lifestyle changes among the residents. 

Other questions were created that may require more study outside of our project's scope. One is surrounding the role of mental health on one's likelihood of binge drinking and alcoholism. Whether these factors also influence
alcohol-related mortality rate may cause Iowa to start implementing related changes to their laws and policies to address this challenge. 
Another area of investigation is surrounding Iowa resident's access to programs or facilities for recovering from alcoholism, as well as the quality of these resources. Regional liquor sales may reveal where residents buy less alcohol, leading to a potential area of study surrounding what neighborhood influences cause consumers to buy less alcohol. Similarly, a final open-ended question is whether there are any major societal trends over the years that impacted alcohol sales.

\section{Methodology}
For the exact methodologies used, it really depends on what specifically you are trying to accomplish with the data. However, a common trend within many of the studies is to break down alcohol consumption by demographics. This article from Pennsylvania State University is a great example of this \cite{penn_state_extension}. While this article is mainly concerned with the types of alcoholic drinks people are consuming, the data is first broken down by age groups. This methodology will be key within our research, since our goal is to figure out how alcohol sales trends relate to lifestyle changes. It is a reasonable assumption that different age groups will make different lifestyle choices. We can then further extend that concept to understand how different geographical regions, ethnicity, and gender influence alcohol choices within certain age cohorts. 

For the analytical part of this project, we want to see what patterns there are within the sales transactions of alcoholic beverages. Regression analysis is always a viable method to observe some overall trends. We plan to use cluster analysis to get a deeper look at how various groups behave when it comes to purchasing alcohol. From there, we will perform further analysis on those sup-groups. The data we are using contains extensive transaction detail, ranging from the liquor store's zip code to the volume of alcohol sold in each transaction. We also plan on incorporating some deep learning methods in order to try and find more complex relationships that regression won't be able to find.

For the visual presentation, we will be using tableau. It is a very powerful visualization package that will allow us to leverage the geographic data present within the data set as well as the specifics for each transaction. Here we will also merge in the Census data in order to get the demographics of the various counties within Iowa \cite{census}.

\section{Conclusion}

While alcohol consumption has a significant social role, the negative effects that alcoholic binge drinking has on a society makes liquor sales and alcohol consumption a topic that is worth delving deeper into. The state of Iowa is a ideal candidate to study due to the detailed information they have available regarding their state's liquor sales. Since the state has been adding in different policies and methods regarding alcohol sales, this study also provides a good opportunity to see if their public policy strategy has had the its intended affect on the issue. Furthermore, researching into the  demographics of the residents of Iowa and their most popular liquor purchases will help create a body of work establishing which factors lead to significant alcohol purchases. This can further help in public health studies addressing binge drinking. Most importantly, to have a successful method of regulating alcohol and ensuring public health, it is critical to understand resident sentiments towards alcohol, and the role that alcohol plays with their lifestyles. Any changing attitudes towards the role alcohol has in social and economic settings is critical for developing any policy related to alcohol sales. 
    
\bibliographystyle{abbrv}
\bibliography{refs}

\end{document}
